\section{Introduction}

iTestClient is used as a client program.
It loads the test created by the server and randomly chooses the questions to form a unique test for the student.
It can either connect to the server over a network or load the test from a file exported by the server.
When the test is over, the results are saved into a file and if the client is connected over network, they are automatically sent to the server.
In case of connection breakdown, the backup file can be used -- it can be loaded back into the server.

\section{On-line usage}

First you will need to get a server running.
See the iTestServer documentation for further details.
You should also know either the IP address of the server or its network name.
For each operating system, there is different way to obtain this information, and it is thus beyond the scope of this documentation.

When the server is running, run iTestClient on a client computer connected to one network with the server.
Type the IP address of the server or its name into the Server name field and by pressing Return (Enter) move to the next field.
Type the server port in here and press Return (Enter) again.
The client should now connect.

\sshot{itestwri-1_0_2-main-network}{Connecting to an iTest server}

\section{Setting the output file location}

The output file contains the answers of the student.
It is used in case of connection breakdown or in case there is no network available and the off-line approach is used.

\sshot{itestwri-1_0_2-main-output}{Setting the output file location}

The default location is the home directory of the current user.
Under Windows, this is the \\Documents and Settings\\UserName directory.

If you need to specify another location, uncheck the Use default checkbox and click Browse.
You shouldn't choose any directory visible to the student, least of all the desktop or the My Documents folder.

\section{Other settings}

\sshot{itestwri-1_1_1-other_settings-hide_question_names}{Hide question names}

If you do not want the students to see the names of the questions, check \quo{Hide question names, show numbers instead}.
This feature was introduced in iTest 1.1.

\sshot{itestclient-1_3_0-other_settings-do_not_show_correct_answers}{Do not show correct answers}

If you do not want correct answers to be displayed at the end of the test, check \quo{Do not show correct answers at the end of the test}.
This feature was introduced in iTest 1.3.

\section{Testing}

When you have configured the client, it is time to click the Ready button in the lower right corner of the screen.

Now is the time for the students to enter the classroom.
They sit down in front of a client computer and are presented with the Welcome screen.
They type in their name and click Start.

They may choose to view the remaining time by clicking Show remaining time in the bottom of the screen.
Each time they select an answer, the corresponding question turns green in the list of questions.
The green colour does not indicate a correct answer -- just an answer.
Whether they're right or wrong, they will be informed at the end of the test (unless you are using the off-line approach, which does not support this).

If a certain question contains SVG graphics, it will be displayed below the text of the question.
Clicking the blue title of a picture shows it in a separate window, allowing the student to resize it and thus take a closer look at it.
This feature was introduced in iTest 1.3.

When they run out of time, the test is automatically ended.
In case they are ready sooner than that, they can choose to finish the exam by clicking Finish.
They will be asked to confirm this action.

\sshot{itestclient-1_2_0-newtest_button}{New test button}

When the test is over, you can start a new one by clicking the New test button.
If you are using the off-line approach, the media with the test file must still be connected.
This feature was introduced in iTest 1.2.
